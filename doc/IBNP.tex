% generated by GAPDoc2LaTeX from XML source (Frank Luebeck)
\documentclass[a4paper,11pt]{report}

            \usepackage[all]{xy} 
        
\usepackage[top=37mm,bottom=37mm,left=27mm,right=27mm]{geometry}
\sloppy
\pagestyle{myheadings}
\usepackage{amssymb}
\usepackage[utf8]{inputenc}
\usepackage{makeidx}
\makeindex
\usepackage{color}
\definecolor{FireBrick}{rgb}{0.5812,0.0074,0.0083}
\definecolor{RoyalBlue}{rgb}{0.0236,0.0894,0.6179}
\definecolor{RoyalGreen}{rgb}{0.0236,0.6179,0.0894}
\definecolor{RoyalRed}{rgb}{0.6179,0.0236,0.0894}
\definecolor{LightBlue}{rgb}{0.8544,0.9511,1.0000}
\definecolor{Black}{rgb}{0.0,0.0,0.0}

\definecolor{linkColor}{rgb}{0.0,0.0,0.554}
\definecolor{citeColor}{rgb}{0.0,0.0,0.554}
\definecolor{fileColor}{rgb}{0.0,0.0,0.554}
\definecolor{urlColor}{rgb}{0.0,0.0,0.554}
\definecolor{promptColor}{rgb}{0.0,0.0,0.589}
\definecolor{brkpromptColor}{rgb}{0.589,0.0,0.0}
\definecolor{gapinputColor}{rgb}{0.589,0.0,0.0}
\definecolor{gapoutputColor}{rgb}{0.0,0.0,0.0}

%%  for a long time these were red and blue by default,
%%  now black, but keep variables to overwrite
\definecolor{FuncColor}{rgb}{0.0,0.0,0.0}
%% strange name because of pdflatex bug:
\definecolor{Chapter }{rgb}{0.0,0.0,0.0}
\definecolor{DarkOlive}{rgb}{0.1047,0.2412,0.0064}


\usepackage{fancyvrb}

\usepackage{mathptmx,helvet}
\usepackage[T1]{fontenc}
\usepackage{textcomp}


\usepackage[
            pdftex=true,
            bookmarks=true,        
            a4paper=true,
            pdftitle={Written with GAPDoc},
            pdfcreator={LaTeX with hyperref package / GAPDoc},
            colorlinks=true,
            backref=page,
            breaklinks=true,
            linkcolor=linkColor,
            citecolor=citeColor,
            filecolor=fileColor,
            urlcolor=urlColor,
            pdfpagemode={UseNone}, 
           ]{hyperref}

\newcommand{\maintitlesize}{\fontsize{50}{55}\selectfont}

% write page numbers to a .pnr log file for online help
\newwrite\pagenrlog
\immediate\openout\pagenrlog =\jobname.pnr
\immediate\write\pagenrlog{PAGENRS := [}
\newcommand{\logpage}[1]{\protect\write\pagenrlog{#1, \thepage,}}
%% were never documented, give conflicts with some additional packages

\newcommand{\GAP}{\textsf{GAP}}

%% nicer description environments, allows long labels
\usepackage{enumitem}
\setdescription{style=nextline}

%% depth of toc
\setcounter{tocdepth}{1}





%% command for ColorPrompt style examples
\newcommand{\gapprompt}[1]{\color{promptColor}{\bfseries #1}}
\newcommand{\gapbrkprompt}[1]{\color{brkpromptColor}{\bfseries #1}}
\newcommand{\gapinput}[1]{\color{gapinputColor}{#1}}


\begin{document}

\logpage{[ 0, 0, 0 ]}
\begin{titlepage}
\mbox{}\vfill

\begin{center}{\maintitlesize \textbf{ IBNP \mbox{}}}\\
\vfill

\hypersetup{pdftitle= IBNP }
\markright{\scriptsize \mbox{}\hfill  IBNP  \hfill\mbox{}}
{\Huge \textbf{ Involutive Bases for Noncommutative Polynomials \mbox{}}}\\
\vfill

{\Huge  0.11 \mbox{}}\\[1cm]
{ 18 September 2024 \mbox{}}\\[1cm]
\mbox{}\\[2cm]
{\Large \textbf{ Gareth A. Evans\\
   \mbox{}}}\\
{\Large \textbf{ Christopher D. Wensley\\
   \mbox{}}}\\
\hypersetup{pdfauthor= Gareth A. Evans\\
   ;  Christopher D. Wensley\\
   }
\end{center}\vfill

\mbox{}\\
{\mbox{}\\
\small \noindent \textbf{ Gareth A. Evans\\
   }  Email: \href{mailto://gareth@mathemateg.com} {\texttt{gareth@mathemateg.com}}\\
  Address: \begin{minipage}[t]{8cm}\noindent
 Ysgol y Creuddyn \\
 Ffordd Derwen, Bae Penrhyn \\
 Llandudno, LL30 3LB \\
 U.K.\\
 \end{minipage}
}\\
{\mbox{}\\
\small \noindent \textbf{ Christopher D. Wensley\\
   }  Email: \href{mailto://cdwensley.maths@btinternet.com} {\texttt{cdwensley.maths@btinternet.com}}\\
  Homepage: \href{https://github.com/cdwensley} {\texttt{https://github.com/cdwensley}}}\\
\end{titlepage}

\newpage\setcounter{page}{2}
{\small 
\section*{Abstract}
\logpage{[ 0, 0, 1 ]}
 The \textsf{IBNP} package provides ... 

Bug reports, comments, suggestions for additional features, and offers to
implement some of these, will all be very welcome. 

Please submit any issues at \href{https://github.com/gap-packages/ibnp/issues/} {\texttt{https://github.com/gap\texttt{\symbol{45}}packages/ibnp/issues/}} or send an email to the second author at \href{mailto://cdwensley.maths@btinternet.com} {\texttt{cdwensley.maths@btinternet.com}}. 

 \mbox{}}\\[1cm]
{\small 
\section*{Copyright}
\logpage{[ 0, 0, 2 ]}
 {\copyright} 2024, Gareth Evans and Chris Wensley.

 The \textsf{IBNP} package is free software; you can redistribute it and/or modify it under the
terms of the GNU General Public License as published by the Free Software
Foundation; either version 2 of the License, or (at your option) any later
version. \mbox{}}\\[1cm]
{\small 
\section*{Acknowledgements}
\logpage{[ 0, 0, 3 ]}
 This documentation was prepared with the \textsf{GAPDoc} \cite{GAPDoc} and \textsf{AutoDoc} \cite{AutoDoc} packages.

 The procedure used to produce new releases uses the package \textsf{GitHubPagesForGAP} \cite{GitHubPagesForGAP} and the package \textsf{ReleaseTools}.

 \mbox{}}\\[1cm]
\newpage

\def\contentsname{Contents\logpage{[ 0, 0, 4 ]}}

\tableofcontents
\newpage

         
\chapter{\textcolor{Chapter }{Introduction}}\label{chap-intro}
\logpage{[ 1, 0, 0 ]}
\hyperdef{L}{X7DFB63A97E67C0A1}{}
{
  The \textsf{IBNP} package provides methods for computing an involutive (Gr{\"o}bner) basis $B$ for an ideal $J$ over a polynomial ring $\mathcal{R}$ in both the commutative and noncommutative cases. Secondly, methods are
provided to involutively reduce a given polynomial to its normal form in $\mathcal{R}/J$. 

 This package was first submitted to run with \textsf{GAP} 4.13.1. 

 The package is loaded with the command 
\begin{Verbatim}[commandchars=!@|,fontsize=\small,frame=single,label=Example]
  
  !gapprompt@gap>| !gapinput@LoadPackage( "ibnp" ); |
  
\end{Verbatim}
 

 The package may be obtained either as a compressed \texttt{.tar} file or as a \texttt{.zip} file, \texttt{ibnp\texttt{\symbol{45}}version{\textunderscore}number.tar.gz}, by ftp from one of the following sites: 
\begin{itemize}
\item  the \textsf{IBNP} GitHub release site: \href{https://gap-packages.github.io/ibnp/} {\texttt{https://gap\texttt{\symbol{45}}packages.github.io/ibnp/}}. 
\item  any \textsf{GAP} archive, e.g. \href{https://www.gap-system.org/Packages/packages.html} {\texttt{https://www.gap\texttt{\symbol{45}}system.org/Packages/packages.html}}; 
\end{itemize}
 \index{GitHub repository} The package also has a GitHub repository at: \href{https://github.com/gap-packages/ibnp} {\texttt{https://github.com/gap\texttt{\symbol{45}}packages/ibnp}}. 

 Once the package is loaded, the manual \texttt{doc/manual.pdf} can be found in the documentation folder. The \texttt{html} versions, with or without MathJax, may be rebuilt as follows: 

 
\begin{Verbatim}[commandchars=!@|,fontsize=\small,frame=single,label=Example]
  
  !gapprompt@gap>| !gapinput@ReadPackage( "ibnp", "makedoc.g" ); |
  
\end{Verbatim}
 

 It is possible to check that the package has been installed correctly by
running the test files (this terminates the \textsf{GAP} session): 

 
\begin{Verbatim}[commandchars=!@|,fontsize=\small,frame=single,label=Example]
  
  !gapprompt@gap>| !gapinput@ReadPackage( "ibnp", "tst/testall.g" );|
  Architecture: . . . . . 
  testing: . . . . . 
  . . . 
  #I  No errors detected while testing
  
\end{Verbatim}
 

 The main reference for this work is Evans' thesis \cite{gareth-thesis}. The main concepts and results may be found in the papers \cite{Beaumont} and \cite{EW-JSC}. }

         
\chapter{\textcolor{Chapter }{Using the packages \textsf{GBNP} and \textsf{NMO}}}\label{chap-desc}
\logpage{[ 2, 0, 0 ]}
\hyperdef{L}{X86057870803DCA93}{}
{
  
\section{\textcolor{Chapter }{Non\texttt{\symbol{45}}commutative polynomials (NPs)}}\label{sec-nps}
\logpage{[ 2, 1, 0 ]}
\hyperdef{L}{X7FDF3E5E7F33D3A2}{}
{
  Recall that the main datatype used by the \textsf{GBNP} package is a list of non\texttt{\symbol{45}}commutative polynomials (NPs). The
data type for a non\texttt{\symbol{45}}commutative polynomial (its NP format)
is a list of two lists: 
\begin{itemize}
\item  The first list is a list $m$ of monomials. 
\item  The second list is a list $c$ of coefficients of these monomials. 
\end{itemize}
 The two lists have the same length. The polynomial represented by the ordered
pair $[m,c]$ is $\sum_i c_i m_i$. A monomial is a list of positive integers. They are interpreted as the
indices of the variables. So, if $k = [1,3,2,2,1]$ and the variables are $a,b,c$ (in this order), then $k$ represents the monomial $acb^2a$. There are various ways to print these but the default uses $a,b,c,\ldots$. The zero polynomial is represented by \texttt{[[],[]]}. The polynomial $1$ is represented by \texttt{[[[]],[1]]}. The algorithms work for the algebra $\mathbb F\langle\langle x_1,x_2,\ldots,x_t\rangle\rangle$ of non\texttt{\symbol{45}}commutative polynomials in \mbox{\texttt{\mdseries\slshape t}} variables over the field $\mathbb F$. Accordingly, the list $c$ should contain elements of $\mathbb F$. 

 \index{NP2GP} \index{GP2NP} The \textsf{GBNP} functions \texttt{GP2NP} and \texttt{NP2GP} convert a polynomial to NP format and back again. Polynomials returned by \texttt{NP2GP} print with their coefficients enclosed in brackets. Polynomials may also be
printed using the function \texttt{PrintNP}. The function PrintNPList is used to print a list of NPs, with one polynomial
per line. The function \texttt{CleanNP} is used to collect terms and reorder them. The default ordering is first by
degree and then lexicographic \texttt{\symbol{45}} \texttt{MonomialGrlexOrdering}. Alternative orderings are available \texttt{\symbol{45}} see section \ref{sec-orderings}. 

 
\begin{Verbatim}[commandchars=!@|,fontsize=\small,frame=single,label=Example]
  
  !gapprompt@gap>| !gapinput@A3 := FreeAssociativeAlgebraWithOne(Rationals,"a","b","c");;|
  !gapprompt@gap>| !gapinput@a := A3.1;; b := A3.2;; c := A3.3;;|
  !gapprompt@gap>| !gapinput@## define a polynomial and convert to NP-format|
  !gapprompt@gap>| !gapinput@p1 := 7*a^2*b*c + 8*b*c*a;|
  (8)*b*c*a+(7)*a^2*b*c
  !gapprompt@gap>| !gapinput@Lp1 := GP2NP( p1 );|
  [ [ [ 1, 1, 2, 3 ], [ 2, 3, 1 ] ], [ 7, 8 ] ]
  !gapprompt@gap>| !gapinput@## define an NP-poly; clean it; and convert to a polynomial|
  !gapprompt@gap>| !gapinput@Lp2 := [ [ [1,1], [1,2,1], [3], [1,1], [3,1,2] ], [5,6,7,6,5] ];;|
  !gapprompt@gap>| !gapinput@PrintNP( Lp2 );|
   5a^2 + 6aba + 7c + 6a^2 + 5cab
  !gapprompt@gap>| !gapinput@Lp2 := CleanNP( Lp2 );|
  [ [ [ 3, 1, 2 ], [ 1, 2, 1 ], [ 1, 1 ], [ 3 ] ], [ 5, 6, 11, 7 ] ]
  !gapprompt@gap>| !gapinput@## note the degree lexicographic ordering|
  !gapprompt@gap>| !gapinput@PrintNP( Lp2 );|
   5cab + 6aba + 11a^2 + 7c
  !gapprompt@gap>| !gapinput@p2 := NP2GP( Lp2, A3 );|
  (5)*c*a*b+(6)*a*b*a+(11)*a^2+(7)*c
  !gapprompt@gap>| !gapinput@PrintNPList( [ Lp1, Lp2, [ [], [] ], [ [ [] ], [9] ] ] );|
   7a^2bc + 8bca
   5cab + 6aba + 11a^2 + 7c
   0
   9 
  
\end{Verbatim}
 }

 
\section{\textcolor{Chapter }{Gr{\"o}bner Bases}}\label{sec-grob}
\logpage{[ 2, 2, 0 ]}
\hyperdef{L}{X7E4277497D877661}{}
{
  The \textsf{GBNP} package computes Gr{\"o}bner bases using the function \texttt{SGrobner}. In the example below the polynomials $\{p,q\}$ define an ideal in $\mathbb Z\langle\langle a,b \rangle\rangle$ which has a three element Gr{\"o}bner basis. 
\begin{Verbatim}[commandchars=!@|,fontsize=\small,frame=single,label=Example]
  
  !gapprompt@gap>| !gapinput@p := [ [ [2,2,2], [2,1], [1,2] ], [1,3,-1] ];;|
  !gapprompt@gap>| !gapinput@q := [ [ [1,1], [2] ], [1,1] ];; |
  !gapprompt@gap>| !gapinput@PrintNPList( [p,q] );|
   b^3 + 3ba - ab
   a^2 + b 
  !gapprompt@gap>| !gapinput@GB := SGrobner( [p,q] );;|
  !gapprompt@gap>| !gapinput@PrintNPList(GB);|
   a^2 + b 
   ba - ab 
   b^3 + 2ab 
  
\end{Verbatim}
 }

 
\section{\textcolor{Chapter }{Orderings for monomials}}\label{sec-orderings}
\logpage{[ 2, 3, 0 ]}
\hyperdef{L}{X7F82A3608248CD31}{}
{
  The three monomial orderings provided by the main library are \texttt{MonomialLexOrdering}, \texttt{MonomialGrlexOrdering} and \texttt{MonomialGrevlexOrdering}. The first of these is the default used by \textsf{GBNP}. 

 The \textsf{NMO} package is now part of the package \textsf{GBNP}. It provides a choice of orderings on monomials, including lexicographic and
length\texttt{\symbol{45}}lexicographic ones. 

 
\begin{Verbatim}[commandchars=!@|,fontsize=\small,frame=single,label=Example]
  
  !gapprompt@gap>| !gapinput@Lp1;|
  [ [ [ 1, 1, 2, 3 ], [ 2, 3, 1 ] ], [ 7, 8 ] ]
  !gapprompt@gap>| !gapinput@Lp2;|
  [ [ [ 3, 1, 2 ], [ 1, 2, 1 ], [ 1, 1 ], [ 3 ] ], [ 5, 6, 11, 7 ] ]
  !gapprompt@gap>| !gapinput@GtNPoly( Lp1, Lp2 );|
  true
  !gapprompt@gap>| !gapinput@## select the lexicographic ordering and reorder p1, p2|
  !gapprompt@gap>| !gapinput@lexord := NCMonomialLeftLexicographicOrdering( A3 );;|
  !gapprompt@gap>| !gapinput@PatchGBNP( lexord );|
  LtNP patched.
  GtNP patched.
  !gapprompt@gap>| !gapinput@Lp1 := CleanNP( Lp1 );|
  [ [ [ 2, 3, 1 ], [ 1, 1, 2, 3 ] ], [ 8, 7 ] ]
  !gapprompt@gap>| !gapinput@Lp2 := CleanNP( Lp2 );|
  [ [ [ 3, 1, 2 ], [ 3 ], [ 1, 2, 1 ], [ 1, 1 ] ], [ 5, 7, 6, 11 ] ]
  !gapprompt@gap>| !gapinput@GtNPoly( Lp1, Lp2 );|
  false
  !gapprompt@gap>| !gapinput@## revert to degree lex order|
  !gapprompt@gap>| !gapinput@UnpatchGBNP();;|
  LtNP restored.
  GtNP restored.
  !gapprompt@gap>| !gapinput@Lp1 := CleanNP( Lp1 );|
  [ [ [ 1, 1, 2, 3 ], [ 2, 3, 1 ] ], [ 7, 8 ] ]
  !gapprompt@gap>| !gapinput@Lp2 := CleanNP( Lp2 );|
  [ [ [ 3, 1, 2 ], [ 1, 2, 1 ], [ 1, 1 ], [ 3 ] ], [ 5, 6, 11, 7 ] ]
  !gapprompt@gap>| !gapinput@GtNPoly( Lp1, Lp2 );|
  true
  
\end{Verbatim}
 }

 }

         
\chapter{\textcolor{Chapter }{Commutative Involutive Bases}}\label{chap-ibases-cp}
\logpage{[ 3, 0, 0 ]}
\hyperdef{L}{X780AAD6F8095AE49}{}
{
  Given a Gr{\"o}bner Basis $G$ for an ideal $J$ over a polynomial ring $\mathcal{R}$, the remainder of any polynomial $p \in \mathcal{R}$ with respect to $G$ is unique. But, although this remainder is unique, there may be many ways of
obtaining this remainder, as it is possible that several polynomials in $G$ divide $p$, giving several \emph{reduction paths} for $p$. 
\section{\textcolor{Chapter }{Reduction Paths}}\label{sec-cib}
\logpage{[ 3, 1, 0 ]}
\hyperdef{L}{X7BBDABD3799443BB}{}
{
  
\subsection{\textcolor{Chapter }{An Example}}\label{subs-ch4ex402}
\logpage{[ 3, 1, 1 ]}
\hyperdef{L}{X7B5623E3821CC0D0}{}
{
  Consider the DegLex Gr{\"o}bner basis $G := \{g_1, g_2, g_3\} = \{a^2-2ab+3, \: 2ab+b^2+5, \:
\frac{5}{4}b^3-\frac{5}{2}a+\frac{37}{4}b\}$ over the polynomial ring $\mathbb{Q}[a,b]$, and consider the polynomial $p := a^2b+b^3+8b$. The remainder of $p$ with respect to $G$ is $0$ (so that $p$ is a member of the ideal $J$ generated by $G$), but there are two ways of obtaining this remainder, as shown in the
following diagram. 
\[ \vcenter{\xymatrix{ & a^2b+b^3+8b \ar[dl]_{g_1} \ar[dr]^{g_2} \\ 2ab^2+b^3+5b
\ar[d]_{g_2} && -\frac{1}{2}ab^2 + b^3 - \frac{5}{2}a+8b \ar[d]^{g_2} \\ 0 &&
\frac{5}{4}b^3-\frac{5}{2}a+\frac{37}{4}b \ar[d]^{g_3} \\ && 0 }} \]
 }

 An \emph{Involutive Basis} for $J$ is a Gr{\"o}bner Basis $G$ such that there is only \emph{one} possible reduction path for any polynomial $p \in \mathcal{R}$. In order to find such a basis, we restrict which reductions or divisions may
take place by requiring, for each potential reduction of a polynomial $p$ by a polynomial $g_i \in G$ (so that $LM(p) = LM(g_i)\times u$ for some monomial $u$), some extra conditions on the variables in $u$ to be satisfied, namely that all variables in $u$ have to be in a set of \emph{multiplicative variables} for $g_i$, a set that is determined by a particular choice of an \emph{involutive division}. }

 
\section{\textcolor{Chapter }{Commutative Involutive Divisions}}\label{sec-invdivc}
\logpage{[ 3, 2, 0 ]}
\hyperdef{L}{X7E43F2087BC8B4F9}{}
{
  Recall that a commutative monomial $u$ is divisible by another monomial $w$ if there exists a third monomial $u'$ such that $u = wu'$. We use the notation $w \mid u$ and refer to $w$ as a \emph{conventional} \index{conventional divisor} divisor of $u$. An involutive division $I$ partitions the variables in the polynomial ring into sets of \emph{multiplicative} and \emph{nonmultiplicative} variables for $u$. The set of multiplicative variables is denoted by $\mathcal{M}_I(u)$. Then $w$ is an \emph{involutive divisor} of $u$, \index{involutive divisor} written $w \mid_I u$, if all variables in $u'$ are in $\mathcal{M}_I(u)$. 
\subsection{\textcolor{Chapter }{Example}}\label{subs-ch4ex411}
\logpage{[ 3, 2, 1 ]}
\hyperdef{L}{X85861B017AEEC50B}{}
{
  Let $u := ab^3c$, $v := abc^3$ and $w := bc$ be three monomials over the polynomial ring $\mathcal{R} := \mathbb{Q}[a,b,c]$. Let an involutive division $I$ partition the variables in $\mathcal{R}$ into the following two sets of variables for the monomial $w$: multiplicative = $\{a,b\}$; nonmultiplicative = $\{c\}$. It is true that $w$ conventionally divides both monomials $u$ and $v$, but $w$ only involutively divides monomial $u$ as, defining $u' := ab^2$ and $v' := ac^2$ (so that $u = wu'$ and $v = wv'$), we observe that all variables in $u'$ are in the set of multiplicative variables for $w$, but the variables in $v'$ (in particular the variable $c$) are not all in the set of multiplicative variables for $w$. So $w \mid_I u$ and $w \nmid_I v$. }

 

\subsection{\textcolor{Chapter }{PommaretDivision}}
\logpage{[ 3, 2, 2 ]}\nobreak
\hyperdef{L}{X82712BA57EBE9170}{}
{\noindent\textcolor{FuncColor}{$\triangleright$\enspace\texttt{PommaretDivision({\mdseries\slshape alg, mons, order})\index{PommaretDivision@\texttt{PommaretDivision}}
\label{PommaretDivision}
}\hfill{\scriptsize (operation)}}\\


 Let $\mathcal{R} = \mathbb{Q}[a_1,\ldots,a_n]$, and let $w$ be a polynomial in $\mathcal{R}$ with leading monomoial $a_1^{e_1}a_2^{e_2} \ldots a_n^{e_n}$ where $e_i$ is the \emph{first} non\texttt{\symbol{45}}zero exponent. The Pommaret involutive division $\mathcal{P}$ sets $\mathcal{M}_{\mathcal{P}}(w) = \{a_1, a_2, \ldots, a_i\}$. 

 Because $\mathcal{M}_{\mathcal{P}}(w)$ does not depend in any way on the other leading monomials in \emph{polys}, this is a \emph{global} division. 

 In The example the first five monomials $u_i$ in $U$ contain a power of $x$, so $\mathcal{M}_{\mathcal{P}}(u_i) = \{x\}$. Then $u_6$ involves $y$ and $z$, so $\mathcal{M}_{\mathcal{P}}(u_6) = \{x,y\}$, and similarly $\mathcal{M}_{\mathcal{P}}(u_7) = \{x,y,z\}$. 

 }

 
\begin{Verbatim}[commandchars=!@|,fontsize=\small,frame=single,label=Example]
  
  !gapprompt@gap>| !gapinput@R := PolynomialRing( Rationals, [ "x", "y", "z" ] );;|
  !gapprompt@gap>| !gapinput@x := R.1;; y := R.2;; z := R.3;;|
  !gapprompt@gap>| !gapinput@U := [ x^5*y^2*z, x^4*y*z^2, x^2*y^2*z, x*y*z^3, x*z^3, y^2*z, z ];|
  [ x^5*y^2*z, x^4*y*z^2, x^2*y^2*z, x*y*z^3, x*z^3, y^2*z, z ]
  !gapprompt@gap>| !gapinput@ord := MonomialLexOrdering();;|
  !gapprompt@gap>| !gapinput@PommaretDivision( R, U, ord );|
  [ [ 1 ], [ 1 ], [ 1 ], [ 1 ], [ 1 ], [ 1, 2 ], [ 1 .. 3 ] ]
  
  
\end{Verbatim}
 

\subsection{\textcolor{Chapter }{ThomasDivision}}
\logpage{[ 3, 2, 3 ]}\nobreak
\hyperdef{L}{X8756720A86A6B125}{}
{\noindent\textcolor{FuncColor}{$\triangleright$\enspace\texttt{ThomasDivision({\mdseries\slshape alg, mons, order})\index{ThomasDivision@\texttt{ThomasDivision}}
\label{ThomasDivision}
}\hfill{\scriptsize (operation)}}\\


 Let $\mathcal{R} = \mathbb{Q}[a_1,\ldots,a_n]$, and let $P$ be a set of polynomials $P = \{p_1,\ldots,p_m\}$ in $\mathcal{R}$ with leading monomials $U = \{u_1,\ldots,u_m\}$ where $u_i = a_1^{e^1_i}a_2^{e^2_i} \ldots a_n^{e^n_i}$. The Thomas involutive division $\mathcal{T}$ sets $a_i$ to be multiplicative for $p_j$ and $u_j$ if $e^i_j = \max_k e^i_k$ for all $1 \leqslant k \leqslant m$. 

 In the example, using the same seven monomials, the highest powers of $[x,y,z]$ are $[5,2,3]$ respectively. So $x$ is multiplicative only for $u_1$, $y$ is multiplicative for $\{u_1,u_3,u_6\}$, and $z$ is multiplicative only for $u_4$ and $u_5$. Note that two of the monomials have no multiplicative variable. 

 }

 
\begin{Verbatim}[commandchars=!@|,fontsize=\small,frame=single,label=Example]
  
  !gapprompt@gap>| !gapinput@ThomasDivision( R, U, ord );|
  [ [ 1, 2 ], [  ], [ 2 ], [ 3 ], [ 3 ], [ 2 ], [  ] ]
  
\end{Verbatim}
 

\subsection{\textcolor{Chapter }{JanetDivision}}
\logpage{[ 3, 2, 4 ]}\nobreak
\hyperdef{L}{X7E214DDF794BB14D}{}
{\noindent\textcolor{FuncColor}{$\triangleright$\enspace\texttt{JanetDivision({\mdseries\slshape alg, mons, order})\index{JanetDivision@\texttt{JanetDivision}}
\label{JanetDivision}
}\hfill{\scriptsize (operation)}}\\


 Let $\mathcal{R} = \mathbb{Q}[a_1,\ldots,a_n]$, and let $P$ be a set of polynomials $P = \{p_1,\ldots,p_m\}$ in $\mathcal{R}$ with leading monomials $U = \{u_1,\ldots,u_m\}$ where $u_i = a_1^{e^1_i}a_2^{e^2_i} \ldots a_n^{e^n_i}$. The Janet involutive division $\mathcal{J}$ sets $a_n$ to be multiplicative for $u_j$ provided $e^n_j = \max_k e^n_k$ for all $1 \leqslant k \leqslant m$. To determine whether $a_i$ is multiplicative for $u_j$, let $L = [e^{i+1}_j,e^{i+2}_j,\ldots,e^n_j]$. Let $S$ be the subset of $\{1,\ldots,m\}$ containing those $k$ such that $[e^{i+1}_k,e^{i+2}_k,\ldots,e^n_k] = L$. Then $a_i$ is multiplicative for $u_j$ provided $e^i_j = \max_{k \in S}e^i_k$. 

 In the example, recall that the exponent lists for the seven monomials are 
\[ [5,2,1],~~~ [4,1,2],~~~ [2,2,1],~~~ [1,1,3],~~~ [1,0,3],~~~ [0,2,1],~~~
[0,0,1]. \]
 As with the Thomas division, $\max_k e^3_k = 3$ and $z$ is multiplicative only for $u_4$ and $u_5$. 

 For $y$, $L = [1]$ when $k \in \{1,3,6,7\}$ and $\max_{\{1,3,6,7\}} e^2_k = 2$, so $y$ is multiplicative for $u_1, u_3$ and $u_6$, but not for $u_7$. $L = [2]$ only for $u_2$, so $y$ is multiplicative for $u_2$. $L = [3]$ for $u_4$ and $u_5$, and $e^2_4 > e^2_5$, so $y$ is multiplicative for $u_4$ but not $u_5$. 

 For $x$, $L = [2,1]$ for $k \in \{1,3,6\}$ and $e^1_1 = 5$ is greater than $e^1_3$ and $e^1_6$, so $x$ is multiplicative for $u_1$. The other values for $L$, namely $[1,2], [1,3], [0,3]$ and $[0,1]$, occur just once each, so $x$ is multiplicative for $u_2, u_4, u_5$ and $u_7$. 

 }

 
\begin{Verbatim}[commandchars=!@|,fontsize=\small,frame=single,label=Example]
  
  !gapprompt@gap>| !gapinput@JanetDivision( R, U, ord );|
  [ [ 1, 2 ], [ 1, 2 ], [ 2 ], [ 1, 2, 3 ], [ 1, 3 ], [ 2 ], [ 1 ] ]
  
  
\end{Verbatim}
 
\subsection{\textcolor{Chapter }{Selecting a Division}}\label{subs-select-divcp}
\logpage{[ 3, 2, 5 ]}
\hyperdef{L}{X83A3B3F77C712DA1}{}
{
  \index{CommutativeDivision} The global variable \texttt{CommutativeDivision} is a string which can take values "Pommaret", "Thomas" or "Janet". The default
is "Pommaret". The example shows how to select the Janet division. }

 
\begin{Verbatim}[commandchars=!@|,fontsize=\small,frame=single,label=Example]
  
  !gapprompt@gap>| !gapinput@CommutativeDivision := "Janet";|
  "Janet"
  
\end{Verbatim}
 

\subsection{\textcolor{Chapter }{DivisionRecord}}
\logpage{[ 3, 2, 6 ]}\nobreak
\hyperdef{L}{X8781FDB7865FA48B}{}
{\noindent\textcolor{FuncColor}{$\triangleright$\enspace\texttt{DivisionRecord({\mdseries\slshape alg, polys, order})\index{DivisionRecord@\texttt{DivisionRecord}}
\label{DivisionRecord}
}\hfill{\scriptsize (function)}}\\
\noindent\textcolor{FuncColor}{$\triangleright$\enspace\texttt{DivisionRecordCP({\mdseries\slshape alg, polys, order})\index{DivisionRecordCP@\texttt{DivisionRecordCP}}
\label{DivisionRecordCP}
}\hfill{\scriptsize (operation)}}\\


 The global function \texttt{DivisionRecord} calls one of the operations \texttt{DivisionRecordCP} and \texttt{DivisionRecordNP}, depending on whether the algebra is commutative or not. In the commutative
case, this function finds the sets of multiplicative variables for a set of
polynomials using one of the involutive divisions listed above. 

 In the following example, polynomials $\{u = b^3-a, v=a^3-b\}$ define an ideal and form a Gr{\"o}bner basis for that ideal. Multiplicative
variables for $u$ are $\{a,b\}$, while those for $v$ are just $\{a\}$. The variable \texttt{drec2.mvars} in the listing below contains the \emph{positions} of these variables in the generating set $\{a,b\}$. }

 
\begin{Verbatim}[commandchars=!@|,fontsize=\small,frame=single,label=Example]
  
  !gapprompt@gap>| !gapinput@R := PolynomialRing( Rationals, [ "a", "b" ] );;|
  !gapprompt@gap>| !gapinput@a := R.1;; b := R.2;;|
  !gapprompt@gap>| !gapinput@L2 := [ b^3 - 3*a, a^3 - 3*b ];;|
  !gapprompt@gap>| !gapinput@ord := MonomialGrlexOrdering();;|
  !gapprompt@gap>| !gapinput@GB2 := ReducedGroebnerBasis( L2, ord );;|
  !gapprompt@gap>| !gapinput@GB2 = L2;|
  true
  !gapprompt@gap>| !gapinput@CommutativeDivision := "Pommaret";;|
  !gapprompt@gap>| !gapinput@drec2 := DivisionRecordCP( R, L2, ord );|
  rec( div := "Pommaret", mvars := [ [ 1, 2 ], [ 1 ] ], 
    polys := [ b^3-3*a, a^3-3*b ] )
  
\end{Verbatim}
 
\[ \vcenter{\xymatrix@=1em{ b & & & \ar@{-}[ddddrrrr] & \ar@{-}[dddrrr] &
\ar@{-}[ddrr] & \ar@{-}[dr] & & & & b & & & & & & & \\ : \ar[u] \ar@{-}[ur] &
\cdot & \cdot & \cdot \ar@{-}[ddddrrrr] & \cdot & \cdot & \cdot & & & & :
\ar[u] \ar@{-}[ur] & \cdot & \cdot & \cdot & \cdot & \cdot & \cdot & \\ 5
\ar@{-}[u] \ar@{-}[uurr] & \cdot & \cdot & \cdot \ar@{-}[ddddrrrr] & \cdot &
\cdot & \cdot & & & & 5 \ar@{-}[u] \ar@{-}[uurr] & \cdot & \cdot & \cdot &
\cdot & \cdot & \cdot & \\ 4 \ar@{-}[u] \ar@{-}[uuurrr] & \cdot & \cdot &
\cdot \ar@{-}[ddddrrrr] & \cdot & \cdot & \cdot & & & & 4 \ar@{-}[u]
\ar@{-}[uuurrr] & \cdot & \cdot & \cdot & \cdot & \cdot & \cdot & \\ \bullet
\ar@{-}[u] \ar@{-}[rrrrrrr] \ar@{-}[uuuurrrr] & \cdot \ar@{-}[uuuurrrr] &
\cdot \ar@{-}[uuuurrrr] & \cdot \ar@{-}[dddrrr] \ar@{-}[uuuurrrr] & \cdot
\ar@{-}[uuurrr] & \cdot \ar@{-}[uurr] & \cdot \ar@{-}[ur] & & & & \bullet
\ar@{-}[u] \ar@{-}[rrrrrrr] \ar@{-}[uuuurrrr] & \cdot \ar@{-}[uuuurrrr] &
\cdot \ar@{-}[uuuurrrr] & \cdot \ar@{-}[uuuurrrr] & \cdot \ar@{-}[uuurrr] &
\cdot \ar@{-}[uurr] & \cdot \ar@{-}[ur] & & \\ 2 \ar@{-}[u] & \cdot & \cdot &
\cdot \ar@{-}[ddrr] & \cdot & \cdot & \cdot & & & & 2 \ar@{-}[u] & \cdot &
\cdot & \bullet \ar@{-}[rrrr]& \cdot & \cdot & \cdot & \\ 1 \ar@{-}[u] & \cdot
& \cdot & \cdot \ar@{-}[dr] & \cdot & \cdot & \cdot & & & & 1 \ar@{-}[u] &
\cdot & \cdot & \bullet \ar@{-}[rrrr] & \cdot & \cdot & \cdot & \\ 0
\ar@{-}[r]\ar@{-}[u] & 1 \ar@{-}[r] & 2 \ar@{-}[r] & \bullet \ar@{-}[r]
\ar@{-}[uuuuuuu] & 4 \ar@{-}[r] & 5 \ar@{-}[r] & \cdots \ar[r] & a & & & 0
\ar@{-}[r]\ar@{-}[u] & 1 \ar@{-}[r] & 2 \ar@{-}[r] & \bullet \ar@{-}[r]
\ar@{-}[uuuuuuu] & 4 \ar@{-}[r] & 5 \ar@{-}[r] & \cdots \ar[r] & a }} \]
 In the \emph{reduction diagrams} above the nodes $(j,k)$ represent the monomials $a^jb^k$. The lead monomials of $u$ and $v$ are marked by bullets. In the left hand diagram the two shaded areas indicate
those monomials which are conventionally reducible by $u$ and by $v$, so that the doubly shaded area contains those monomials which are
conventionally reducible by both. For an involutive division, this must be
avoided. 

 In the right\texttt{\symbol{45}}hand diagram we see that $u$ involutively divides the same set of monomials in the main shaded area. On the
other hand $v$ just involutively divides monomials $\{a^j \mid j \ge 3\}$. So none of the monomials $\{a^jb, a^jb^2 \mid j \ge 3\}$ reduce by $v$ involutively. The operation \texttt{InvolutiveBasis}, to be described below, produces two further polynomials, $w = vb = a^3b-3b^2$ and $vb^2$ which reduces by $u$ to $x = a^3b^2 - 9a$. Both $w$ and $x$ have multiplicative variables $\{a\}$ and the monomials which they can reduce lie on the two horizantal line
segments in the right\texttt{\symbol{45}}hand diagram. In this way, all the
conventionally reducible monomials are involutively reducible by just one of $\{u,v,w,x\}$. 

 The polynomial $p = a^3b^3 + 2a^3b + 3ab^3$ reduces involutively as follows. 
\[ p \stackrel{u}{\longrightarrow} 3a^4 + 2a^3b + 3ab^3
\stackrel{v}{\longrightarrow} 2a^3b + 3ab^3 + 9ab
\stackrel{w}{\longrightarrow} 3ab^3 + 9ab + 6b^2 \stackrel{u}{\longrightarrow}
9a^2 + 9ab + 6b^2 \]
 This reduction is computed in section \ref{InvolutiveBasis}. 

\subsection{\textcolor{Chapter }{IPolyReduce}}
\logpage{[ 3, 2, 7 ]}\nobreak
\hyperdef{L}{X79F5892C80AE2667}{}
{\noindent\textcolor{FuncColor}{$\triangleright$\enspace\texttt{IPolyReduce({\mdseries\slshape algebra, polynomial, DivisionRecord, order})\index{IPolyReduce@\texttt{IPolyReduce}}
\label{IPolyReduce}
}\hfill{\scriptsize (function)}}\\
\noindent\textcolor{FuncColor}{$\triangleright$\enspace\texttt{IPolyReduceCP({\mdseries\slshape algebra, polynomial, DivisionRecord, order})\index{IPolyReduceCP@\texttt{IPolyReduceCP}}
\label{IPolyReduceCP}
}\hfill{\scriptsize (operation)}}\\


 The global function \texttt{IPolyReduce} calls one of the operations \texttt{IPolyReduceCP} and \texttt{IPolyReduceNP} depending on whether the algebra is commutative or not. This function reduces
a polynomial $p$ using the current overlap record for a basis, and an ordering. 

 In the example the polynomial $p$ reduces only to $9a^2+9ab+2a^3b$ because the polynomial $x$ is not available to reduce $2a^3b$ to $6b^2$. }

 
\begin{Verbatim}[commandchars=!@|,fontsize=\small,frame=single,label=Example]
  
  !gapprompt@gap>| !gapinput@p := a^3*b^3 + 2*a^3*b + 3*a*b^3;;|
  !gapprompt@gap>| !gapinput@q := IPolyReduce( R, p, drec2, ord );|
  2*a^3*b+9*a^2+9*a*b
  
\end{Verbatim}
 

\subsection{\textcolor{Chapter }{IAutoreduce}}
\logpage{[ 3, 2, 8 ]}\nobreak
\hyperdef{L}{X7C58A339832877E9}{}
{\noindent\textcolor{FuncColor}{$\triangleright$\enspace\texttt{IAutoreduce({\mdseries\slshape alg, polys, order})\index{IAutoreduce@\texttt{IAutoreduce}}
\label{IAutoreduce}
}\hfill{\scriptsize (function)}}\\
\noindent\textcolor{FuncColor}{$\triangleright$\enspace\texttt{IAutoreduceCP({\mdseries\slshape alg, polys, order})\index{IAutoreduceCP@\texttt{IAutoreduceCP}}
\label{IAutoreduceCP}
}\hfill{\scriptsize (operation)}}\\


 The global function \texttt{IAutoreduce} calls one of the operations \texttt{IAutoreduceCP} and \texttt{IAutoreduceNP} depending on whether the algebra is commutative or not. This function applies \texttt{IPolyReduce} to a list of polynomials recursively until no more reductions are possible.
More specifically, this function involutively reduces each member of a list of
polynomials with respect to all the other members of the list, removing the
polynomial from the list if it is involutively reduced to 0. This process is
iterated until no more reductions are possible. 

 In the example we form \texttt{L} by adding $p$ to \texttt{L2}. Applying \texttt{IAutoreduceCP} only $p$ reduces, and the concatenation of \texttt{L2} with $q$ is returned. 

 }

 
\begin{Verbatim}[commandchars=!@|,fontsize=\small,frame=single,label=Example]
  
  !gapprompt@gap>| !gapinput@L := Concatenation( L2, [p] );;|
  !gapprompt@gap>| !gapinput@IAutoreduceCP( R, L, ord );|
  [ b^3-3*a, a^3-3*b, 2*a^3*b+9*a^2+9*a*b ]
  
\end{Verbatim}
 }

 
\section{\textcolor{Chapter }{Computing a Commutative Involutive Basis}}\label{sec-compibc}
\logpage{[ 3, 3, 0 ]}
\hyperdef{L}{X864907F987701716}{}
{
  The involutive algorithm for constructing an involutive basis uses \emph{prolongations} and \emph{autoreduction}. 
\subsection{\textcolor{Chapter }{Prolongations and Autoreduction}}\label{subs-prolong}
\logpage{[ 3, 3, 1 ]}
\hyperdef{L}{X7ACAA0847CC0DBCC}{}
{
  Given a set of polynomials $P$, a \emph{prolongation} of $ p \in P$ is a product $pa_i$ where the generator $a_i$ is \emph{not} multiplicative with respect to the current involutive division. 

 A set of polynomials $P$ is said to be \emph{autoreduced} if no polynomial $p \in P$ contains a term which is involutively divisible by some polynomial $p' \in P \setminus \{p\}$. 

 We denote by $rem_I(p,Q)$ the involutive remainder of polynomial $p$ with respect to a set of polynomials $Q$. Here is the \emph{Commutative Autoreduction Algorithm}: 
\begin{Verbatim}[commandchars=!@|,fontsize=\small,frame=single,label=]
  Input: a set of polynomials P = {p_1,p_2,...,p_n} and an involutive division I 
    while there exists p_i in P such that rem_I(p_i, P\{p_i}) <> p_i do
      q := Rem_I(p_i, P\{p_i});
      P := P\{p_i};
      if (q<>0) then
        P := P union {q};
      fi;
    od;
    return P;
\end{Verbatim}
 

 It can be shown that if $P$ is a set of polynomials over a polynomial ring $\mathcal{R} = R[a_1,\ldots,a_n]$, such that $P$ is autoreduced with respect to an involutive division $I$, and if $p,q$ are two polynomials in $\mathcal{R}$, then $rem_I(p,P) + rem_I(q,P) = rem_I(p+q,P)$. 

 Given an involutive division $I$ and an admissible monomial ordering $O$, an autoreduced set of polynomials $P$ is a \emph{locally involutive basis} with respect to $I$ and $O$ if any prolongation of any $p_i \in P$ involutively reduces to zero using $P$. Further, $P$ is an \emph{involutive basis} with respect to $I$ and $O$ if any multiple $p_it$ of any $p_i \in P$ by any term $t$ involutively reduces to zero using $P$. 

 The \emph{Commutative Involutive Basis Algorithm}: 
\begin{Verbatim}[commandchars=!@|,fontsize=\small,frame=single,label=]
  Input: a basis F = {f_1,f_2,...,f_m} for an ideal J 
         over a commutative polynomial ring R[a_1,...,a_n];
         an admissible monomial ordering O; 
         a continuous and constructive involutive division I 
  Output: an involutive basis G = {g_1,g_2,...,g_k} for J (if it terminates)
    G := { };
    F := autoreduction of F with respect to O and I;
    while G = { } do
      P := set of all prolongations f_i*a_j, 1<=i<=m, 1<=j<=n;
      q := 0;
      while (P <> { }) and (q=0) do
        p := a polynomial in P with minimal lead monomial w.r.t. O;
        P := P \ {p};
        q := rem_I(p,F);
      od;
      if (q <> 0) then  ## new basis element found
        F := autoreduction of (F union {q})
      else  ## all the prolongations have reduced to zero
        G := F;
      fi;
    od;
    return G;
\end{Verbatim}
 

 }

 

\subsection{\textcolor{Chapter }{InvolutiveBasis}}
\logpage{[ 3, 3, 2 ]}\nobreak
\hyperdef{L}{X7B60A306820D4ED2}{}
{\noindent\textcolor{FuncColor}{$\triangleright$\enspace\texttt{InvolutiveBasis({\mdseries\slshape alg, polys, order})\index{InvolutiveBasis@\texttt{InvolutiveBasis}}
\label{InvolutiveBasis}
}\hfill{\scriptsize (function)}}\\
\noindent\textcolor{FuncColor}{$\triangleright$\enspace\texttt{InvolutiveBasisCP({\mdseries\slshape alg, polys, order})\index{InvolutiveBasisCP@\texttt{InvolutiveBasisCP}}
\label{InvolutiveBasisCP}
}\hfill{\scriptsize (operation)}}\\


 The global function \texttt{InvolutiveBasis} calls one of the operations \texttt{InvolutiveBasisCP} and \texttt{InvolutiveBasisNP} depending on whether the algebra is commutative or not. This function finds an
involutive basis for the ideal generated by a set of polynomials, using a
chosen ordering, and returns a division record. 

 Any involutive basis returned by this algorithm is a Gr{\"o}bner basis, and
remainders are involutively unique with respect to this basis. 

 }

 
\begin{Verbatim}[commandchars=!@|,fontsize=\small,frame=single,label=Example]
  
  !gapprompt@gap>| !gapinput@ibasP := InvolutiveBasis( R, L2, ord );|
  rec( div := "Pommaret", mvars := [ [ 1, 2 ], [ 1 ], [ 1 ], [ 1 ] ], 
    polys := [ b^3-3*a, a^3-3*b, a^3*b-3*b^2, a^3*b^2-9*a ] )
  !gapprompt@gap>| !gapinput@r := IPolyReduce( R, p, ibasP, ord );|
  9*a^2+9*a*b+6*b^2
  
\end{Verbatim}
 In the example above we start with $F=\{u,v\}$. There is only one prolongation, $P=\{w=vb\}$ and $F$ becomes $\{u,v,w\}$. At the second iteration, $P=\{vb,wb\}$; $vb$ reduces to zero; $wb$ reduces to $x=a^3b^2-9a$; and $F$ becomes $\{u,v,w,x\}$. At the third iteration $P=\{vb,wb,xb\}$ and all three of these reduce to zero, so $G = F$ is returned. 

 It is then shown that the multiplicative variables for $\{u,v,w,x\}$ are $\{\{a,b\},\{a\},\{a\},\{a\}\}$. 

 Finally, the full reduction $r$ of $p$, as described at the end of section \ref{DivisionRecord}, is computed. 

 If, instead of the Pommaret division, we use Thomas or Janet we obtain: 
\begin{Verbatim}[commandchars=!@|,fontsize=\small,frame=single,label=Example]
  
  !gapprompt@gap>| !gapinput@CommutativeDivision := "Thomas";;|
  !gapprompt@gap>| !gapinput@ibasT := InvolutiveBasis( R, L2, ord );|
  rec( div := "Thomas", 
    mvars := [ [ 2 ], [ 1 ], [ 2 ], [ 1 ], [ 2 ], [ 1 ], [ 1, 2 ] ], 
    polys := [ b^3-3*a, a^3-3*b, a*b^3-3*a^2, a^3*b-3*b^2, a^2*b^3-9*b, 
        a^3*b^2-9*a, a^3*b^3-9*a*b ] )
  !gapprompt@gap>| !gapinput@CommutativeDivision := "Janet";;|
  !gapprompt@gap>| !gapinput@ibasJ := InvolutiveBasis( R, L2, ord );;|
  !gapprompt@gap>| !gapinput@ibasJ.mvars = ibasP.mvars;|
  true
  !gapprompt@gap>| !gapinput@ibasJ.polys = ibasP.polys;|
  true
  
\end{Verbatim}
 The Janet division gives the same involutive basis as Pommaret, but the Thomas
Division produces $7$, rather than $4$ polynomials: 
\[ [~ b^3-3a,\ a^3-3b,\ ab^3-3a^2,\ a^3b-3b^2,\ a^2b^3-9b,\ a^3b^2-9a,\
a^3b^3-9ab ~]. \]
 

 The multiplicative variables for these polynomials are $[ \{b\}, \{a\}, \{b\}, \{a\}, \{b\}, \{a\}, \{a,b\} ]$, so the reduction diagram for the Thomas basis is: 
\[ \vcenter{\xymatrix@=1em{ b & & & & & & & & \\ : \ar[u] & \cdot & \cdot & \cdot
\ar@{-}[ur] & \cdot & \cdot & \cdot & & \\ 5 \ar@{-}[u] & \cdot & \cdot &
\cdot \ar@{-}[uurr] & \cdot & \cdot & \cdot & & \\ 4 \ar@{-}[u] & \cdot &
\cdot & \cdot \ar@{-}[uuurrr] & \cdot & \cdot & \cdot & & \\ \bullet
\ar@{-}[u] \ar@{-}[rrrrrrr] & \bullet \ar@{-}[uuuu] & \bullet \ar@{-}[uuuu] &
\bullet \ar@{-}[uuuurrrr] & \cdot \ar@{-}[uuurrr] & \cdot \ar@{-}[uurr] &
\cdot \ar@{-}[ur] & & \\ 2 \ar@{-}[u] & \cdot & \cdot & \bullet \ar@{-}[rrrr]
& \cdot & \cdot & \cdot & & \\ 1 \ar@{-}[u] & \cdot & \cdot & \bullet
\ar@{-}[rrrr] & \cdot & \cdot & \cdot & & \\ 0 \ar@{-}[r]\ar@{-}[u] & 1
\ar@{-}[r] & 2 \ar@{-}[r] & \bullet \ar@{-}[r] \ar@{-}[uuuuuuu] & 4 \ar@{-}[r]
& 5 \ar@{-}[r] & \cdots \ar[r] & a & }} \]
 }

 }

         
\chapter{\textcolor{Chapter }{Functions for Noncommutative Monomials}}\label{chap-monom}
\logpage{[ 4, 0, 0 ]}
\hyperdef{L}{X872783907DFA29B7}{}
{
  A word, such as $ab^2a$ is represented in \textsf{GBNP} as the list $[1,2,2,1]$. Polynomials have a more complicated structure, for example $6ab^2a - 7ab + 8ba$ is represented in \textsf{GBNP} by $[ [ [1,2,2,1], [1,2], [2,1] ], [6,-7,8] ]$, which is a list of monomials followed by the corresponding list of
coefficients. Polynomials are dealt with in the following chapter. 

 As shown in Scetion \ref{sec-nps}, \textsf{GBNP} has functions \texttt{PrintNP} and \texttt{PrintNPList} to print a polynomial and a list of polynomials. Here we provide equivalent
functions for monomials. 
\section{\textcolor{Chapter }{Basic functions for monomials}}\label{sec-basic}
\logpage{[ 4, 1, 0 ]}
\hyperdef{L}{X846C3B0F79265278}{}
{
  

\subsection{\textcolor{Chapter }{PrintNM}}
\logpage{[ 4, 1, 1 ]}\nobreak
\hyperdef{L}{X7D53D8657AEDFEB2}{}
{\noindent\textcolor{FuncColor}{$\triangleright$\enspace\texttt{PrintNM({\mdseries\slshape monomial})\index{PrintNM@\texttt{PrintNM}}
\label{PrintNM}
}\hfill{\scriptsize (operation)}}\\
\noindent\textcolor{FuncColor}{$\triangleright$\enspace\texttt{PrintNMList({\mdseries\slshape list})\index{PrintNMList@\texttt{PrintNMList}}
\label{PrintNMList}
}\hfill{\scriptsize (operation)}}\\


 Recall, from \textsf{GBNP}, that the actual letters printed are controlled by the operation \texttt{GBNP.ConfigPrint}. }

 
\begin{Verbatim}[commandchars=!@|,fontsize=\small,frame=single,label=Example]
  
  !gapprompt@gap>| !gapinput@GBNP.ConfigPrint( "a", "b" );|
  !gapprompt@gap>| !gapinput@mon := [2,1,1,1,2,2,1];;|
  !gapprompt@gap>| !gapinput@PrintNM( mon );|
  ba^3b^2a
  !gapprompt@gap>| !gapinput@L := [ [1,2,2], [2,1,2], [1,1,1], [2], [ ] ];;|
  !gapprompt@gap>| !gapinput@PrintNMList( L );                            |
  ab^2
  bab
  a^3
  b
  1
  
\end{Verbatim}
 

\subsection{\textcolor{Chapter }{PrefixNM}}
\logpage{[ 4, 1, 2 ]}\nobreak
\hyperdef{L}{X7F72641C8441204E}{}
{\noindent\textcolor{FuncColor}{$\triangleright$\enspace\texttt{PrefixNM({\mdseries\slshape monomial, posint})\index{PrefixNM@\texttt{PrefixNM}}
\label{PrefixNM}
}\hfill{\scriptsize (operation)}}\\
\noindent\textcolor{FuncColor}{$\triangleright$\enspace\texttt{SubwordNM({\mdseries\slshape monomial, posint, posint})\index{SubwordNM@\texttt{SubwordNM}}
\label{SubwordNM}
}\hfill{\scriptsize (operation)}}\\
\noindent\textcolor{FuncColor}{$\triangleright$\enspace\texttt{SuffixNM({\mdseries\slshape monomial, posint})\index{SuffixNM@\texttt{SuffixNM}}
\label{SuffixNM}
}\hfill{\scriptsize (operation)}}\\


 These are the three operations which pick a sublist from a monomial list. }

 
\begin{Verbatim}[commandchars=!@|,fontsize=\small,frame=single,label=Example]
  
  !gapprompt@gap>| !gapinput@mon := [2,1,1,1,2,2,1];;|
  !gapprompt@gap>| !gapinput@PrefixNM( mon, 3 );|
  [ 2, 1, 1 ]
  !gapprompt@gap>| !gapinput@SubwordNM( mon, 3, 6 );|
  [ 1, 1, 2, 2 ]
  !gapprompt@gap>| !gapinput@SuffixNM( mon, 3 );|
  [ 2, 2, 1 ]
  
\end{Verbatim}
 

\subsection{\textcolor{Chapter }{SuffixPrefixPosNM}}
\logpage{[ 4, 1, 3 ]}\nobreak
\hyperdef{L}{X8046DF397ACA0E5E}{}
{\noindent\textcolor{FuncColor}{$\triangleright$\enspace\texttt{SuffixPrefixPosNM({\mdseries\slshape monomial, monomial, posint, posint})\index{SuffixPrefixPosNM@\texttt{SuffixPrefixPosNM}}
\label{SuffixPrefixPosNM}
}\hfill{\scriptsize (operation)}}\\


 The operation \texttt{SuffixPrefixPosNM( left, right, start, limit)} looks for overlaps of type \emph{suffix of left = prefix of right}. The size of the smallest such overlap is returned. Which overlaps are
considered is controlled by the third and fourth arguments. We commence by
looking at the overlap of size \emph{start} and go no further than the overlap of size \emph{limit}. When no overlap exists, $0$ is returned. To test all possibilities, \emph{start} should be $1$ and \emph{limit} should be $min(|left|,|right|)-1$. It is the user's responsibility to make sure that these bounds are correct
\texttt{\symbol{45}} no checks are made. 

 }

 
\begin{Verbatim}[commandchars=!@|,fontsize=\small,frame=single,label=Example]
  
  !gapprompt@gap>| !gapinput@m1 := [2,1,1,1,2,2,1,1];;           ## m1 = ba^3b^2a^2|
  !gapprompt@gap>| !gapinput@m2 := [1,1,2,2,1,1];;               ## m2 = a^2b^2a^2|
  !gapprompt@gap>| !gapinput@SuffixPrefixPosNM( m1, m2, 1, 5 );  ## overlap is a                   |
  1
  !gapprompt@gap>| !gapinput@SuffixPrefixPosNM( m1, m2, 2, 5 );  ## overlap is a^2|
  2
  !gapprompt@gap>| !gapinput@SuffixPrefixPosNM( m1, m2, 3, 5 );  ## no longer an overlap|
  0
  !gapprompt@gap>| !gapinput@SuffixPrefixPosNM( m2, m1, 1, 5 );  ## overlap is ba^2|
  3
  
\end{Verbatim}
 

\subsection{\textcolor{Chapter }{SubwordPosNM}}
\logpage{[ 4, 1, 4 ]}\nobreak
\hyperdef{L}{X82916CB37D346978}{}
{\noindent\textcolor{FuncColor}{$\triangleright$\enspace\texttt{SubwordPosNM({\mdseries\slshape monomial, monomial, posint})\index{SubwordPosNM@\texttt{SubwordPosNM}}
\label{SubwordPosNM}
}\hfill{\scriptsize (operation)}}\\
\noindent\textcolor{FuncColor}{$\triangleright$\enspace\texttt{IsSubwordNM({\mdseries\slshape monomial, monomial})\index{IsSubwordNM@\texttt{IsSubwordNM}}
\label{IsSubwordNM}
}\hfill{\scriptsize (operation)}}\\


 The operation \texttt{SubwordPosNM( small, large, start );} answers the question for monomials \emph{Is small a subword of large?}. The value returned is the start position in \emph{large} of the first subword found. When no subword is found, $0$ is returned. The search commences at position \emph{start} in \emph{large} so, to test all possibilities, the third argument should be $1$. 

 To just ask whether or not \emph{small} is a subword of \emph{large}, just use \texttt{IsSubwordNM( small, large);}. }

 
\begin{Verbatim}[commandchars=!@|,fontsize=\small,frame=single,label=Example]
  
  !gapprompt@gap>| !gapinput@m3 := [ 1, 1, 2 ];;|
  !gapprompt@gap>| !gapinput@SubwordPosNM( m3, m1, 1 );|
  3
  !gapprompt@gap>| !gapinput@SubwordPosNM( m3, m2, 1 );|
  1
  !gapprompt@gap>| !gapinput@SubwordPosNM( m3, m2, 2 );|
  0
  !gapprompt@gap>| !gapinput@IsSubwordNM( [ 2, 1, 2 ], m1 );|
  false
  
\end{Verbatim}
 

\subsection{\textcolor{Chapter }{LeadVarNM}}
\logpage{[ 4, 1, 5 ]}\nobreak
\hyperdef{L}{X83CF80DD7CD5F166}{}
{\noindent\textcolor{FuncColor}{$\triangleright$\enspace\texttt{LeadVarNM({\mdseries\slshape monomial})\index{LeadVarNM@\texttt{LeadVarNM}}
\label{LeadVarNM}
}\hfill{\scriptsize (operation)}}\\
\noindent\textcolor{FuncColor}{$\triangleright$\enspace\texttt{LeadExpNM({\mdseries\slshape monomial})\index{LeadExpNM@\texttt{LeadExpNM}}
\label{LeadExpNM}
}\hfill{\scriptsize (operation)}}\\
\noindent\textcolor{FuncColor}{$\triangleright$\enspace\texttt{TailNM({\mdseries\slshape monomial})\index{TailNM@\texttt{TailNM}}
\label{TailNM}
}\hfill{\scriptsize (operation)}}\\


 Given the word $w = b^4a^3c^2$, represented by $[2,2,2,2,1,1,1,3,3]$, the \emph{lead variable} is $b$ or $2$, and the \emph{lead exponent} is $4$. Removing $b^4$ from $w$ leaves the \emph{tail} $a^3c^2$. }

 
\begin{Verbatim}[commandchars=!@|,fontsize=\small,frame=single,label=Example]
  
  !gapprompt@gap>| !gapinput@mon := [2,2,2,2,1,1,1,3,3];;|
  !gapprompt@gap>| !gapinput@LeadVarNM( mon );           |
  2
  !gapprompt@gap>| !gapinput@LeadExpNM( mon );           |
  4
  !gapprompt@gap>| !gapinput@TailNM( mon );           |
  [ 1, 1, 1, 3, 3 ]
  
\end{Verbatim}
 

\subsection{\textcolor{Chapter }{DivNM}}
\logpage{[ 4, 1, 6 ]}\nobreak
\hyperdef{L}{X7CECFE0C86895946}{}
{\noindent\textcolor{FuncColor}{$\triangleright$\enspace\texttt{DivNM({\mdseries\slshape monomial, monomial})\index{DivNM@\texttt{DivNM}}
\label{DivNM}
}\hfill{\scriptsize (operation)}}\\


 The operation \texttt{DivNM( large, small);} for two monomials returns all the ways that \emph{small} divides \emph{large} in the form of a list of pairs of monomials \emph{[left,right]} so that \emph{large = left*small*right}. In the example we search for subwords $ab$ of $m = abcababc$, returning $[ [abcab,c], [abc,abc], [1,cababc] ]$. }

 
\begin{Verbatim}[commandchars=!@|,fontsize=\small,frame=single,label=Example]
  
  !gapprompt@gap>| !gapinput@m := [ 1, 2, 3, 1, 2, 1, 2, 3 ];;|
  !gapprompt@gap>| !gapinput@d := [ 1, 2 ];;|
  !gapprompt@gap>| !gapinput@PrintNMList( [ m, d ] );|
  abcababc
  ab                
  !gapprompt@gap>| !gapinput@divs := DivNM( m, d ); |
  [ [ [ 1, 2, 3, 1, 2 ], [ 3 ] ], [ [ 1, 2, 3 ], [ 1, 2, 3 ] ], 
    [ [  ], [ 3, 1, 2, 1, 2, 3 ] ] ]
  !gapprompt@gap>| !gapinput@PrintNMList( divs[1] );|
  abcab
  c
  
\end{Verbatim}
 }

 }

         
\chapter{\textcolor{Chapter }{Functions for Noncommutative Polynomials}}\label{chap-poly}
\logpage{[ 5, 0, 0 ]}
\hyperdef{L}{X7BD27C5585EF8629}{}
{
  A word, 
\section{\textcolor{Chapter }{Basic functions for polynomials}}\label{sec-polyops}
\logpage{[ 5, 1, 0 ]}
\hyperdef{L}{X80FC94957D03EEA6}{}
{
  
\subsection{\textcolor{Chapter }{Predefined algebras}}\label{sub-inbuilt-alg}
\logpage{[ 5, 1, 1 ]}
\hyperdef{L}{X84F106BB8093FCAE}{}
{
  For convenience of use in examples, three algebras over the rationals, \texttt{AlbebraIBNP} and \texttt{AlgebrakIBNP} with $k \in [2,3,4]$, are predefined by the package. 
\begin{Verbatim}[commandchars=!@|,fontsize=\small,frame=single,label=Example]
  
  !gapprompt@gap>| !gapinput@GeneratorsOfAlgebra( Algebra2IBNP );|
  [ (1)*<identity ...>, (1)*a, (1)*b ]
  !gapprompt@gap>| !gapinput@GeneratorsOfAlgebra( Algebra3IBNP );|
  [ (1)*<identity ...>, (1)*a, (1)*b, (1)*c ]
  !gapprompt@gap>| !gapinput@GeneratorsOfAlgebra( Algebra4IBNP );|
  [ (1)*<identity ...>, (1)*a, (1)*A, (1)*b, (1)*B ]
  !gapprompt@gap>| !gapinput@AlgebraIBNP = Algebra2IBNP;|
  true
  
\end{Verbatim}
 }

 

\subsection{\textcolor{Chapter }{MaxDegreeNP}}
\logpage{[ 5, 1, 2 ]}\nobreak
\hyperdef{L}{X7A1E54F279CCCF65}{}
{\noindent\textcolor{FuncColor}{$\triangleright$\enspace\texttt{MaxDegreeNP({\mdseries\slshape polylist})\index{MaxDegreeNP@\texttt{MaxDegreeNP}}
\label{MaxDegreeNP}
}\hfill{\scriptsize (operation)}}\\


 Given an \texttt{FAlgList}, this function calculates the degree of the lead term for each element of the
list and returns the largest value found. }

 
\begin{Verbatim}[commandchars=!@|,fontsize=\small,frame=single,label=Example]
  
  !gapprompt@gap>| !gapinput@A2 := AlgebraIBNP;|
  <algebra-with-one over Rationals, with 2 generators>
  !gapprompt@gap>| !gapinput@a := A2.1;; b := A2.2;;|
  !gapprompt@gap>| !gapinput@ord := NCMonomialLeftLengthLexicographicOrdering( A2 );;|
  !gapprompt@gap>| !gapinput@t := [ [ [1,2,1,1,2,1], [2,2,1,2], [2,1,1,2] ], [6,7,8] ];;|
  !gapprompt@gap>| !gapinput@u := [ [ [1,1,2,1], [1,2,2], [2,1] ], [4,-2,1] ];;|
  !gapprompt@gap>| !gapinput@v := [ [ [2,1,2], [1,2], [2] ], [2,-1,3] ];; |
  !gapprompt@gap>| !gapinput@w := [ [ [1,1,2], [2,1], [1] ], [3,2,-1] ];;|
  !gapprompt@gap>| !gapinput@L4 := [ t, u, v, w ];; |
  !gapprompt@gap>| !gapinput@PrintNPList( L4 );|
   6aba^2ba + 7b^2ab + 8ba^2b 
   4a^2ba - 2ab^2 + ba 
   2bab - ab + 3b 
   3a^2b + 2ba - a 
  !gapprompt@gap>| !gapinput@MaxDegreeNP( L4 );|
  6
  
\end{Verbatim}
 

\subsection{\textcolor{Chapter }{MyScalarMulNP}}
\logpage{[ 5, 1, 3 ]}\nobreak
\hyperdef{L}{X855F79E07BE46C7D}{}
{\noindent\textcolor{FuncColor}{$\triangleright$\enspace\texttt{MyScalarMulNP({\mdseries\slshape pol, const})\index{MyScalarMulNP@\texttt{MyScalarMulNP}}
\label{MyScalarMulNP}
}\hfill{\scriptsize (operation)}}\\


 Arithmetic with polynomials is performed using the \textsf{GBNP} functions \texttt{AddNP}, \texttt{MulNP} and \texttt{BiMulNP}. We find it convenient to add here a function which multiplies a polynomial
by an element of the underlying field of the algebra. 

 Does this actually get used? 

 Better to move it to \textsf{GBNP}? }

 
\begin{Verbatim}[commandchars=!@|,fontsize=\small,frame=single,label=Example]
  
  !gapprompt@gap>| !gapinput@w2 := MyScalarMulNP( w, 2 );;|
  !gapprompt@gap>| !gapinput@PrintNP( w2 );|
   6a^2b + 4ba - 2a
  
\end{Verbatim}
 

\subsection{\textcolor{Chapter }{LtNPoly}}
\logpage{[ 5, 1, 4 ]}\nobreak
\hyperdef{L}{X7996395279064998}{}
{\noindent\textcolor{FuncColor}{$\triangleright$\enspace\texttt{LtNPoly({\mdseries\slshape pol1, pol2})\index{LtNPoly@\texttt{LtNPoly}}
\label{LtNPoly}
}\hfill{\scriptsize (operation)}}\\
\noindent\textcolor{FuncColor}{$\triangleright$\enspace\texttt{GtNPoly({\mdseries\slshape pol1, pol2})\index{GtNPoly@\texttt{GtNPoly}}
\label{GtNPoly}
}\hfill{\scriptsize (operation)}}\\


 These two functions generalise the \textsf{GBNP} functions \texttt{LtNP} and \texttt{GtNP} which (confusingly) apply only to monomials. They compare a pair of
polynomials with respect to the monomial ordering currently being used. In the
example we check that $v > w$, that $w < 2w$ and $u < u+ba$. }

 
\begin{Verbatim}[commandchars=!@|,fontsize=\small,frame=single,label=Example]
  
  !gapprompt@gap>| !gapinput@LtNPoly( v, w );|
  false
  !gapprompt@gap>| !gapinput@LtNPoly( w, w2 );     |
  true
  !gapprompt@gap>| !gapinput@u2 := AddNP( u, [ [ [2,1] ], [1] ], 1, 1 );;|
  !gapprompt@gap>| !gapinput@PrintNPList( [u,u2] );|
   4a^2ba - 2ab^2 + ba 
   4a^2ba - 2ab^2 + 2ba 
  !gapprompt@gap>| !gapinput@LtNPoly( u, u2 );|
  true
  !gapprompt@gap>| !gapinput@## LtNPoly and GtNPoly may be used within the Sort command:|
  !gapprompt@gap>| !gapinput@L5 := [u,v,w,u2,w2];|
  [ [ [ [ 1, 1, 2, 1 ], [ 1, 2, 2 ], [ 2, 1 ] ], [ 4, -2, 1 ] ], 
    [ [ [ 2, 1, 2 ], [ 1, 2 ], [ 2 ] ], [ 2, -1, 3 ] ], 
    [ [ [ 1, 1, 2 ], [ 2, 1 ], [ 1 ] ], [ 3, 2, -1 ] ], 
    [ [ [ 1, 1, 2, 1 ], [ 1, 2, 2 ], [ 2, 1 ] ], [ 4, -2, 2 ] ], 
    [ [ [ 1, 1, 2 ], [ 2, 1 ], [ 1 ] ], [ 6, 4, -2 ] ] ]
  !gapprompt@gap>| !gapinput@Sort( L5, GtNPoly );|
  !gapprompt@gap>| !gapinput@L5;|
  [ [ [ [ 1, 1, 2, 1 ], [ 1, 2, 2 ], [ 2, 1 ] ], [ 4, -2, 2 ] ], 
    [ [ [ 1, 1, 2, 1 ], [ 1, 2, 2 ], [ 2, 1 ] ], [ 4, -2, 1 ] ], 
    [ [ [ 2, 1, 2 ], [ 1, 2 ], [ 2 ] ], [ 2, -1, 3 ] ], 
    [ [ [ 1, 1, 2 ], [ 2, 1 ], [ 1 ] ], [ 6, 4, -2 ] ], 
    [ [ [ 1, 1, 2 ], [ 2, 1 ], [ 1 ] ], [ 3, 2, -1 ] ] ]
  
\end{Verbatim}
 

\subsection{\textcolor{Chapter }{LowestLeadMonomialPosNP}}
\logpage{[ 5, 1, 5 ]}\nobreak
\hyperdef{L}{X79B2E02082C8799E}{}
{\noindent\textcolor{FuncColor}{$\triangleright$\enspace\texttt{LowestLeadMonomialPosNP({\mdseries\slshape polylist})\index{LowestLeadMonomialPosNP@\texttt{LowestLeadMonomialPosNP}}
\label{LowestLeadMonomialPosNP}
}\hfill{\scriptsize (operation)}}\\


 Given a list of polynomials, this function looks at all the leading monomials
and returns the position of the smallest lead monomial with respect to the
monomial ordering currently being used. In the example, since \texttt{L5} is sorted, the fifth polynomial is the least. }

 
\begin{Verbatim}[commandchars=!@|,fontsize=\small,frame=single,label=Example]
  
  !gapprompt@gap>| !gapinput@LowestLeadMonomialPosNP( L5 );|
  5
  
\end{Verbatim}
 }

 }

         
\chapter{\textcolor{Chapter }{Noncommutative Involutive Bases}}\label{chap-ibases-np}
\logpage{[ 6, 0, 0 ]}
\hyperdef{L}{X797E483A84214975}{}
{
  When applying a noncommutative rewriting system we conventionally apply a rule $\ell \to r$ to a word $w$ if and only if $w$ has the form $w = u \ell v$, where $u$ or $v$ may be the empty word $\epsilon$. Then $w$ reduces to $urv$. 

 An \emph{involutive monoid rewriting system} $I$ will restrict these conventional reductions by imposing a limitation on the
letters allowed in $u$ and $v$. Sets $\mathcal{M}^L_I(w)$, the \emph{left multiplicative variables} for $w$, and $\mathcal{M}^R_I(w)$, the \emph{right multiplicative variables} for $w$, are defined by $I$. 
\section{\textcolor{Chapter }{Noncommutative Involutive Divisions}}\label{sec-invdivn}
\logpage{[ 6, 1, 0 ]}
\hyperdef{L}{X7A86C2437F6EB83D}{}
{
  An \emph{involutive division} $\mathcal{I}$ is a procedure for determining, given an arbitrary set of monomials $W$, sets of left and right multiplicative letters $\mathcal{M}^L_I(\ell,W)$ and $\mathcal{M}^R_I(\ell,W)$ for any $\ell \in W$. Then set $\mathcal{M}^L_I(W) = \{\mathcal{M}^L_I(\ell,W) \mid \ell \in W\}$ and $\mathcal{M}^R_I(W) = \{\mathcal{M}^R_I(\ell,W) \mid \ell \in W\}$. 

 An \emph{involutive rewriting system} $I$ is \emph{based on $\mathcal{I}$} if $\mathcal{M}^L_I(W)$ and $\mathcal{M}^R_I(W)$ are determined using $\mathcal{I}$, in which case we may write $\mathcal{M}^L_{\mathcal{I}}(W)$ and $\mathcal{M}^R_{\mathcal{I}}(W)$ for these sets of letters. 

 A word $\ell$ is an \emph{involutive divisor} of $w$, written $\ell \mid_I w$, if 
\begin{itemize}
\item  $w = u \ell v$; 
\item  either $u = \epsilon$, or the last letter of $u$ is left multiplicative for $\ell$; 
\item  and either $v = \epsilon$, or the first letter of $v$ is right multiplicative for $\ell$. 
\end{itemize}
 When this is the case, $w$ \emph{involutively reduces} to $urv$ by the rule $\ell \to r$. 

 For example, let $M = rws(\{x,y,z\},~ \{xy \to z,~ yz \to x\})$, so that $W = \{xy,yz\}$. Choose left and right multiplicative variables as shown in the following
table: \begin{center}
\begin{tabular}{|c|c|c|}\hline
$\ell$&
$\mathcal{M}^L_I(\ell,W)$&
$\mathcal{M}^R_I(\ell,W)$\\
\hline
$xy$&
$\{x,y,z\}$&
$\{y,z\}$\\
$yz$&
$\{y,z\}$&
$\{x\}$\\
\hline
\end{tabular}\\[2mm]
\end{center}

 We consider reductions of $w = xyzx$. Conventionally, both rules may be used., giving reductions $z^2x$ and $x^3$ respectively. Involutively, we see that $xy \mid_I xyzx$ because $z$ is right multiplicative for $xy$, but $yz \not{\mid}_I~ xyzx$ because $x$ is left nonmultiplkiucative for $yz$. Thus the only involutive reduction is $xyzx \to_I z^2x$. 

 If an involutive division $\mathcal{I}$ determines the left and right multiplicative variables for a word $\ell in W$ \emph{independently} of the set $W$, then the division is known as a \emph{global involutive division}. Otherwise $\mathcal{I}$ is a \emph{local involutive division}. 

\subsection{\textcolor{Chapter }{LeftDivision}}
\logpage{[ 6, 1, 1 ]}\nobreak
\hyperdef{L}{X8593BCDB8402C46C}{}
{\noindent\textcolor{FuncColor}{$\triangleright$\enspace\texttt{LeftDivision({\mdseries\slshape alg, mons, order})\index{LeftDivision@\texttt{LeftDivision}}
\label{LeftDivision}
}\hfill{\scriptsize (operation)}}\\


 Given a word $w$, the \emph{left division} $\triangleleft$ assigns all letters to be left multiplicative for $w$, and all letters to be right nonmultiplicative for $w$. The example is taken from Example 5.5.12 in the thesis \cite{gareth-thesis}. }

 
\begin{Verbatim}[commandchars=!@|,fontsize=\small,frame=single,label=Example]
  
  !gapprompt@gap>| !gapinput@A3 := Algebra3IBNP;;|
  !gapprompt@gap>| !gapinput@a:=A3.1;;  b:=A3.2;; c:=A3.3;;|
  !gapprompt@gap>| !gapinput@ord := NCMonomialLeftLengthLexicographicOrdering( A3 );;|
  !gapprompt@gap>| !gapinput@M6 := [ a*b, a, b*c, a*c, c*b, c^2 ];;           |
  !gapprompt@gap>| !gapinput@U6 := [ [1,2], [1], [2,3], [1,3], [3,2], [3,3] ];;|
  !gapprompt@gap>| !gapinput@LeftDivision( A3, U6, ord );   |
  [ [ [ 1 .. 3 ], [ 1 .. 3 ], [ 1 .. 3 ], [ 1 .. 3 ], [ 1 .. 3 ], [ 1 .. 3 ] ], 
    [ [  ], [  ], [  ], [  ], [  ], [  ] ] ]
  
\end{Verbatim}
 

\subsection{\textcolor{Chapter }{RightDivision}}
\logpage{[ 6, 1, 2 ]}\nobreak
\hyperdef{L}{X784AF6B87B2B5E5D}{}
{\noindent\textcolor{FuncColor}{$\triangleright$\enspace\texttt{RightDivision({\mdseries\slshape alg, mons, order})\index{RightDivision@\texttt{RightDivision}}
\label{RightDivision}
}\hfill{\scriptsize (operation)}}\\


 Given a word $w$, the \emph{right division} $\triangleright$ assigns all letters to be left nonmultiplicative for $w$, and all letters to be right multiplicative for $w$. }

 
\begin{Verbatim}[commandchars=!@|,fontsize=\small,frame=single,label=Example]
  
  !gapprompt@gap>| !gapinput@RightDivision( A3, U6, ord );|
  [ [ [  ], [  ], [  ], [  ], [  ], [  ] ], 
    [ [ 1 .. 3 ], [ 1 .. 3 ], [ 1 .. 3 ], [ 1 .. 3 ], [ 1 .. 3 ], [ 1 .. 3 ] ] ]
  
\end{Verbatim}
 

\subsection{\textcolor{Chapter }{LeftOverlapDivision}}
\logpage{[ 6, 1, 3 ]}\nobreak
\hyperdef{L}{X7A979BF38311024C}{}
{\noindent\textcolor{FuncColor}{$\triangleright$\enspace\texttt{LeftOverlapDivision({\mdseries\slshape alg, mons, order})\index{LeftOverlapDivision@\texttt{LeftOverlapDivision}}
\label{LeftOverlapDivision}
}\hfill{\scriptsize (operation)}}\\


 Let $W = \{w_1, \ldots, w_m\}$. The \emph{left overlap division} $\mathcal{L}$ assumes, to begin with, that all letters are left and right multiplicative for
every $w_i$. It then assigns some letters to be right nonmultiplicative as follows. 
\begin{itemize}
\item  Suppose $w_j \in W$ is a \emph{subword}, but not a suffix, of a (different) word $w_i \in W$. Then, for some $k$, we have $w_j =$ Subword($w_i,k,k+deg(w_j)-1$). Assign the letter in position $k+deg(w_j) \in w_i$ to be right nonmultiplicative for $w_j$. 
\item  Suppose a proper \emph{prefix} of $w_i$ is equal to a proper \emph{suffix} of a (not neccessarily different) $w_j$, and that $w_i$ is not a proper subword of $w_j$, or vice versa. Then, for some $k$, we have Prefix($w_i,k$) = Suffix($w_j,k$). Assign the letter in position $k+1$ in $w_i$ to be right nonmultiplicative for $w_j$. 
\end{itemize}
 Fox example, consider the rewriting system with rules $\{ab^2 \to b,~ ba^2 \to a\}$, so that the leading monomials are $\{u=ab^2, v=ba^2\}$. Neither monomial is a subword of the other, so the first rule above does not
apply. Since Prefix($v,1) = b =$ Suffix($u,1$), then $v[2]=b$ is assigned to be right nonmulitplicative for $u$. By symmetry, $u[2]=a$ is assigned to be right nonmulitplicative for $v$. The resulting sets are shown in the following table. \begin{center}
\begin{tabular}{|c|c|c|}\hline
$w$&
$\mathcal{M}^L_{\mathcal{L}}(w,W)$&
$\mathcal{M}^R_{\mathcal{L}}(w,W)$\\
\hline
$u = ab^2$&
$\{a,b\}$&
$\{a\}$\\
$v = ba^2$&
$\{a,b\}$&
$\{b\}$\\
\hline
\end{tabular}\\[2mm]
\end{center}

 }

 

 The following example continues Example 5.5.12 in the thesis \cite{gareth-thesis}. 
\begin{Verbatim}[commandchars=!@|,fontsize=\small,frame=single,label=Example]
  
  !gapprompt@gap>| !gapinput@LeftOverlapDivision( A3, U6, ord );               |
  [ [ [ 1 .. 3 ], [ 1 .. 3 ], [ 1 .. 3 ], [ 1 .. 3 ], [ 1 .. 3 ], [ 1 .. 3 ] ], 
    [ [ 1, 2 ], [ 1 ], [ 1 ], [ 1 ], [ 1, 2 ], [ 1 ] ] ]
  
\end{Verbatim}
 

\subsection{\textcolor{Chapter }{RightOverlapDivision}}
\logpage{[ 6, 1, 4 ]}\nobreak
\hyperdef{L}{X83CE05CF7CB18611}{}
{\noindent\textcolor{FuncColor}{$\triangleright$\enspace\texttt{RightOverlapDivision({\mdseries\slshape alg, mons, order})\index{RightOverlapDivision@\texttt{RightOverlapDivision}}
\label{RightOverlapDivision}
}\hfill{\scriptsize (operation)}}\\


 This division is the mirror image of \texttt{LeftOverlapDivision}. }

 

 
\begin{Verbatim}[commandchars=!@|,fontsize=\small,frame=single,label=Example]
  
  !gapprompt@gap>| !gapinput@RightOverlapDivision( A3, U6, ord );               |
  [ [ [ 1 .. 3 ], [ 1 .. 3 ], [ 2 ], [ 1 .. 3 ], [  ], [  ] ], 
    [ [ 1 .. 3 ], [ 1 .. 3 ], [ 1 .. 3 ], [ 1 .. 3 ], [ 1 .. 3 ], [ 1 .. 3 ] ] ]
  
\end{Verbatim}
 
\subsection{\textcolor{Chapter }{Selecting a Division}}\label{subs-select-divnp}
\logpage{[ 6, 1, 5 ]}
\hyperdef{L}{X83A3B3F77C712DA1}{}
{
  \index{NoncommutativeDivision} The global variable \texttt{NoncommutativeDivision} which can take values "Left", "Right", "LeftOverlap" or "RightOverlap". The
default is "LeftOverlap". The example shows how to select the left division. }

 
\begin{Verbatim}[commandchars=!@|,fontsize=\small,frame=single,label=Example]
  
  !gapprompt@gap>| !gapinput@NoncommutativeDivision := "LeftOverlap";|
  "LeftOverlap"
  
\end{Verbatim}
 Other divisions may be added in due course. 

\subsection{\textcolor{Chapter }{DivisionRecordNP}}
\logpage{[ 6, 1, 6 ]}\nobreak
\hyperdef{L}{X86FAAD527E20A573}{}
{\noindent\textcolor{FuncColor}{$\triangleright$\enspace\texttt{DivisionRecordNP({\mdseries\slshape alg, mons, order})\index{DivisionRecordNP@\texttt{DivisionRecordNP}}
\label{DivisionRecordNP}
}\hfill{\scriptsize (operation)}}\\


 This operation is called by the global function \texttt{DivisionRecord} when the algebra is noncommutative. 

 This operation finds the sets of multiplicative variables for a set of
polynomials using one of the involutive divisions listed above. As in the
commutative case, a three\texttt{\symbol{45}}field record \texttt{drec} is returned: \texttt{drec.div} is the division string; \texttt{drec.mvars} is a two\texttt{\symbol{45}}element list, the first listing the sets of left
multiplicative variables, and the second listing the sets of right
multiplicative variables; \texttt{drec.polys} is the list of polynomials in NP\texttt{\symbol{45}}format. 

 }

 
\begin{Verbatim}[commandchars=!@|,fontsize=\small,frame=single,label=Example]
  
  !gapprompt@gap>| !gapinput@L3 := [ [ [ [1,2,2], [3] ], [1,-1] ],|
  !gapprompt@>| !gapinput@           [ [ [2,3,3], [1] ], [1,-1] ],|
  !gapprompt@>| !gapinput@           [ [ [3,1,1], [2] ], [1,-1] ] ];;|
  !gapprompt@gap>| !gapinput@PrintNPList( L3 );|
   ab^2 - c 
   bc^2 - a 
   ca^2 - b 
  !gapprompt@gap>| !gapinput@drec := DivisionRecord( A3, L3, ord );|
  rec( div := "LeftOverlap", 
    mvars := [ [ [ 1 .. 3 ], [ 1 .. 3 ], [ 1 .. 3 ] ], 
        [ [ 1, 2 ], [ 2, 3 ], [ 1, 3 ] ] ], 
    polys := [ [ [ [ 1, 2, 2 ], [ 3 ] ], [ 1, -1 ] ], 
        [ [ [ 2, 3, 3 ], [ 1 ] ], [ 1, -1 ] ], 
        [ [ [ 3, 1, 1 ], [ 2 ] ], [ 1, -1 ] ] ] )
  
\end{Verbatim}
 

\subsection{\textcolor{Chapter }{IPolyReduceNP}}
\logpage{[ 6, 1, 7 ]}\nobreak
\hyperdef{L}{X828DA2AE844847E9}{}
{\noindent\textcolor{FuncColor}{$\triangleright$\enspace\texttt{IPolyReduceNP({\mdseries\slshape algebra, polynomial, DivisionRecord, order})\index{IPolyReduceNP@\texttt{IPolyReduceNP}}
\label{IPolyReduceNP}
}\hfill{\scriptsize (operation)}}\\


 This operation is called by the global function \texttt{IPolyReduce} when the algebra is noncommutative. This function reduces a polynomial $p$ using the current overlap record for a basis, and an ordering. 

 In the example $p = 5c^2a^2b^2 + 6b^2c^2a^2 + 7a^2b^2c^2$. The monomial $c^2a^2b^2$ reduces to $c^2ac$ by $ab^2 \to c$, since there are no letters to the right, but not by $ca^2 \to b$ since $ca^2$ is not right multiplicative by $b$. The other terms are simiolar, and $p$ reduces to $5c^2ac + 6b^2cb + 7a^2ba$. }

 
\begin{Verbatim}[commandchars=!@|,fontsize=\small,frame=single,label=Example]
  
  !gapprompt@gap>| !gapinput@## choose a polynomial to reduce|
  !gapprompt@gap>| !gapinput@p := 5*c^2*a^2*b^2 + 6*b^2*c^2*a^2 + 7*a^2*b^2*c^2;;|
  !gapprompt@gap>| !gapinput@## convert to NP format and reduce|
  !gapprompt@gap>| !gapinput@Lp := GP2NP( p );|
  [ [ [ 3, 3, 1, 1, 2, 2 ], [ 2, 2, 3, 3, 1, 1 ], [ 1, 1, 2, 2, 3, 3 ] ], 
    [ 5, 6, 7 ] ]
  !gapprompt@gap>| !gapinput@Lrp := IPolyReduce( A3, Lp, drec, ord );;|
  !gapprompt@gap>| !gapinput@## convert back to a polynomial|
  !gapprompt@gap>| !gapinput@rp := NP2GP( Lrp, A3 );|
  (5)*c^2*a*c+(6)*b^2*c*b+(7)*a^2*b*a
  !gapprompt@gap>| !gapinput@## p-rp should now belong to the ideal and reduce to 0|
  !gapprompt@gap>| !gapinput@q := p - rp;;|
  !gapprompt@gap>| !gapinput@Lq := GP2NP( q );;|
  !gapprompt@gap>| !gapinput@Lrq := IPolyReduce( A3, Lq, drec, ord );;|
  !gapprompt@gap>| !gapinput@rq := NP2GP( Lrq, A3 );|
  <zero> of ...
  
\end{Verbatim}
 

\subsection{\textcolor{Chapter }{IAutoreduceNP}}
\logpage{[ 6, 1, 8 ]}\nobreak
\hyperdef{L}{X8189DEDD87CE1667}{}
{\noindent\textcolor{FuncColor}{$\triangleright$\enspace\texttt{IAutoreduceNP({\mdseries\slshape alg, polys, order})\index{IAutoreduceNP@\texttt{IAutoreduceNP}}
\label{IAutoreduceNP}
}\hfill{\scriptsize (operation)}}\\


 This operation is called by the global function \texttt{IAutoreduce} when the algebra is noncommutative. This function applies \texttt{IPolyReduceNP} to a list of polynomials recursively until no more reductions are possible.
More specifically, this function involutively reduces each member of a list of
polynomials with respect to all the other members of the list, removing the
polynomial from the list if it is involutively reduced to 0. This process is
iterated until no more reductions are possible. 

 In the example we form \texttt{L4} by adding \texttt{Lp} to \texttt{L3}. Applying \texttt{IAutoreduceNP} only $p$ reduces, and the concatenation of \texttt{L3} with $Lrp$ is returned. 

 }

 
\begin{Verbatim}[commandchars=!@|,fontsize=\small,frame=single,label=Example]
  
  !gapprompt@gap>| !gapinput@L4 := Concatenation( L3, [Lp] );;|
  !gapprompt@gap>| !gapinput@R4 := IAutoreduceNP( A3, L4, ord );;|
  !gapprompt@gap>| !gapinput@PrintNPList( R4 );|
   5c^2ac + 6b^2cb + 7a^2ba 
   ca^2 - b 
   bc^2 - a 
   ab^2 - c 
  
\end{Verbatim}
 }

 
\section{\textcolor{Chapter }{Computing a Noncommutative Involutive Basis}}\label{sec-compibn}
\logpage{[ 6, 2, 0 ]}
\hyperdef{L}{X80C3BE018688AFB7}{}
{
  The involutive algorithm for constructing an involutive basis in the
noncommutative case also uses \emph{prolongations} and \emph{autoreduction}. 

\subsection{\textcolor{Chapter }{InvolutiveBasisNP}}
\logpage{[ 6, 2, 1 ]}\nobreak
\hyperdef{L}{X7A71E4CD7B43726B}{}
{\noindent\textcolor{FuncColor}{$\triangleright$\enspace\texttt{InvolutiveBasisNP({\mdseries\slshape alg, polys, order})\index{InvolutiveBasisNP@\texttt{InvolutiveBasisNP}}
\label{InvolutiveBasisNP}
}\hfill{\scriptsize (operation)}}\\


 This operation is called by the global function \texttt{InvolutiveBasis} when the algebra is noncommutative. This function finds an involutive basis
for the ideal generated by a set of polynomials, using a chosen ordering. 

 In the example we find thqt a Gr{\"o}bner basis starting from \texttt{L3} is rather large. so add three more polynomials $[a^2b-c,b^2c-a,c^2a-b]$ defining the ideal. The resulting Gr{\"o}bner basis then has just three terms.
We then calculate an involutive basis, which has just seven terms. We also
find the reduced form of $p$ to be $18a^2$. }

 
\begin{Verbatim}[commandchars=@|D,fontsize=\small,frame=single,label=Example]
  
  @gapprompt|gap>D @gapinput|gbas := SGrobner( L3 );;D
  @gapprompt|gap>D @gapinput|Length( gbas );         D
  64
  @gapprompt|gap>D @gapinput|## that's too large an example to continue with!D
  @gapprompt|gap>D @gapinput|K3 := [ [ [ [1,1,2], [3] ], [1,-1] ],D
  @gapprompt|>D @gapinput|           [ [ [2,2,3], [1] ], [1,-1] ],D
  @gapprompt|>D @gapinput|           [ [ [3,3,1], [2] ], [1,-1] ] ];;D
  @gapprompt|gap>D @gapinput|L6 := Concatenation( L3, K3 );;D
  @gapprompt|gap>D @gapinput|gbas := SGrobner( L6 );;D
  @gapprompt|gap>D @gapinput|PrintNPList( gbas );D
   b - a 
   c - a 
   a^3 - a 
  @gapprompt|gap>D @gapinput|## so the only reduced elements are {1,a,a^2}D
  @gapprompt|gap>D @gapinput|ibas := InvolutiveBasis( A3, L6, ord );D
  rec( div := "LeftOverlap", 
    mvars := [ [ [ 1, 2 ], [ 1, 2 ], [ 1, 2 ], [ 1, 2 ], [ 1, 2 ], [ 1, 2 ] ], 
        [ [  ], [  ], [  ], [ 1 ], [ 2 ], [  ] ] ], 
    polys := [ [ [ [ 2, 1, 1, 2 ], [ 1, 2 ] ], [ 1, -1 ] ], 
        [ [ [ 1, 2, 2, 1 ], [ 2, 1 ] ], [ 1, -1 ] ], 
        [ [ [ 2, 1, 2 ], [ 2 ] ], [ 1, -1 ] ], 
        [ [ [ 2, 1, 1 ], [ 1 ] ], [ 1, -1 ] ], 
        [ [ [ 1, 2, 2 ], [ 2 ] ], [ 1, -1 ] ], 
        [ [ [ 1, 2, 1 ], [ 1 ] ], [ 1, -1 ] ] ] )
  rec( div := "LeftOverlap", 
    mvars := 
      [ 
        [ [ 1 .. 3 ], [ 1 .. 3 ], [ 1 .. 3 ], [ 1 .. 3 ], [ 1 .. 3 ], 
            [ 1 .. 3 ], [ 1 .. 3 ] ], 
        [ [ 2, 3 ], [ 2, 3 ], [ 2, 3 ], [ 2, 3 ], [ 2, 3 ], [ 2, 3 ], [ 2, 3 ] 
           ] ], 
    polys := [ [ [ [ 3, 1, 1 ], [ 1 ] ], [ 1, -1 ] ], 
        [ [ [ 2, 1, 1 ], [ 1 ] ], [ 1, -1 ] ], 
        [ [ [ 1, 1, 1 ], [ 1 ] ], [ 1, -1 ] ], 
        [ [ [ 3, 1 ], [ 1, 1 ] ], [ 1, -1 ] ], 
        [ [ [ 2, 1 ], [ 1, 1 ] ], [ 1, -1 ] ], [ [ [ 3 ], [ 1 ] ], [ 1, -1 ] ], 
        [ [ [ 2 ], [ 1 ] ], [ 1, -1 ] ] ] )
  @gapprompt|gap>D @gapinput|PrintNPList( ibas.polys );             D
   ca^2 - a 
   ba^2 - a 
   a^3 - a 
   ca - a^2 
   ba - a^2 
   c - a 
   b - a 
  @gapprompt|gap>D @gapinput|Lr := IPolyReduce( A3, p, ibas, ord );;D
  @gapprompt|gap>D @gapinput|PrintNP( Lr );D
   18a^2 
  
\end{Verbatim}
 In this simple example the left division produces the same basis, while the
right and right overlap divisions produce (as might be expected) a mirror
image basis. 

 
\begin{Verbatim}[commandchars=!@|,fontsize=\small,frame=single,label=Example]
  
  !gapprompt@gap>| !gapinput@NoncommutativeDivision := "RightOverlap";;|
  !gapprompt@gap>| !gapinput@ibas := InvolutiveBasis( A3, L6, ord );;  |
  !gapprompt@gap>| !gapinput@PrintNPList( ibas.polys );                |
   a^2c - a 
   a^2b - a 
   a^3 - a 
   ac - a^2 
   ab - a^2 
   c - a 
   b - a 
  
\end{Verbatim}
 }

 }

         
\chapter{\textcolor{Chapter }{Examples}}\label{chap-examples}
\logpage{[ 7, 0, 0 ]}
\hyperdef{L}{X7A489A5D79DA9E5C}{}
{
  To be added. }

 \def\bibname{References\logpage{[ "Bib", 0, 0 ]}
\hyperdef{L}{X7A6F98FD85F02BFE}{}
}

\bibliographystyle{alpha}
\bibliography{bib.xml}

\addcontentsline{toc}{chapter}{References}

\def\indexname{Index\logpage{[ "Ind", 0, 0 ]}
\hyperdef{L}{X83A0356F839C696F}{}
}

\cleardoublepage
\phantomsection
\addcontentsline{toc}{chapter}{Index}


\printindex

\immediate\write\pagenrlog{["Ind", 0, 0], \arabic{page},}
\newpage
\immediate\write\pagenrlog{["End"], \arabic{page}];}
\immediate\closeout\pagenrlog
\end{document}
